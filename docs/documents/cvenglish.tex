\documentclass[letterpaper]{article}
\usepackage[utf8x]{inputenc} 
\usepackage[pdftex]{graphicx}
\usepackage{tabularx}
\usepackage{CV}
\usepackage{url}
\usepackage[dvips,final]{epsfig}

\setlength{\topskip}{0cm}
\setlength{\footskip}{0cm}
\setlength{\topmargin}{0cm}
\setlength{\headsep}{0cm}
\begin{document}

\pagestyle{empty}

\begin{center}
\huge{\textsc{Curriculum Vitae}}
\vspace{\baselineskip}
\end{center}


\begin{table}[h]
\begin{center}
\begin{tabular}[b]{>{\hspace{0.9cm}}c@{\hspace{1.3cm}}r}

\Large{\textsc{Felipe José Bravo Márquez}}

\end{tabular}
\end{center}
\end{table}


%\begin{figure}[h!]
%\center
%\includegraphics[height=3cm]{foto_chica.jpg}
%\end{figure}

%\vspace{1.5\baselineskip}

\section{Personal Information}

\begin{flushleft}
  \textbf{Nationality}: Chile. \\
  \textbf{Address}: 4c Gazeley Avenue, Hamilton, New Zealand.\\
  \textbf{Date of Birth}: 3rd of January 1984. \\
  \textbf{Passport Number}: P05083829. \\
  \textbf{Phone Number}: +642-20647146.\\
  \textbf{E-mail}: felipebravom@gmail.com. \\
  \textbf{Web Page}: \url{http://www.cs.waikato.ac.nz/~fbravoma/}. \\
  \textbf{Areas of Interest}: Natural Language Processing, Machine Learning, Data Mining, Information Retrieval, Social Media Analysis.
  
\end{flushleft}

\section{Education}


\begin{CV}


\item[2014-2017] PhD in Computer Science, The University of Waikato, New Zealand.

\item[2011-2013] M.Sc. in Computer Science, University of Chile (summa cum laude).

\item[2010] Professional Degree in Industrial Engineering, University of Chile (summa cum laude).

\item[2010] Professional Degree in Computer Science Engineering, University of Chile (summa cum laude).

\item[2005-2009]  Bachelor of Science in Industrial Engineering, University of Chile.

\item[2003-2008]  Bachelor in Computer Science, University of Chile.



\end{CV}




\section{Work Experience}

\begin{CV}
\item[2017-present] Research Fellow, \textit{Machine Learning Group, University of Waikato}. \\
\item[2011-2013] Research Engineer, \textit{Yahoo! Labs Latin America}. \\ \url{http://labs.yahoo.com/Yahoo_Labs_Santiago} 
\item[2009- 2011] Researcher and Software Developer, \textit{Web Intelligence Consortium Chile Research Centre}.    \\ \url{http://wi.dii.uchile.cl}
\item[2009] Internship, \textit{Simple}, Software Developer. \\ \url{http://www.simple.cl}.
\item[2009] Internship, \textit{Previred}, Business Process Management Area. \\ \url{http://www.previred.com}.
\item[2005-2006] Internship, \textit{Vitanet}, Software Developer. \\ \url{http://www.vitanet.cl}. 

\end{CV}


\begin{flushleft}
\emph{Scholarships}
\end{flushleft}

\begin{CV}

\item[2017] University of Waikato Doctoral Publications Scholarship.

\item[2016] IEEE student travel grant to attend the 2016
IEEE/WIC/ACM International Conference on Web Intelligence in Omaha, Nebraska, USA.

\item[2014-2017]  University of Waikato Doctoral Scholarship.

\item[2011-2012]  National Scientific and Technological Research Commission (CONICYT) Scholarship for master's degree studies at Chilean Universities.



\end{CV}



\section{Publications}

\begin{flushleft}
\emph{Journal Papers}
\end{flushleft}

\begin{enumerate}


\item F. Bravo-Marquez, E. Frank, and B. Pfahringer \textit{Transferring Sentiment Knowledge between Words and Tweets}, In \textit{Web Intelligence}, to appear.

\item F. Bravo-Marquez, E. Frank, and B. Pfahringer \textit{Building a Twitter Opinion Lexicon from Automatically-annotated Tweets}, In \textit{Knowledge-Based Systems}  Volume 108, September 2016, Pages 65–78.

\item J.D. Velásquez, Y. Covacevich, F. Molina, E. Marrese-Taylor, C. Rodríguez, and F. Bravo-Marquez \textit{DOCODE 3.0 (DOcument COpy DEtector): A system for plagiarism detection by applying an information fusion process from multiple documental data sources}, In \textit{Information Fusion} Volume 27, January 2016, Pages 64–75. 


\item E. Marrese-Taylor, J.D. Velásquez, F. Bravo-Marquez \textit{A Novel Deterministic Approach for Aspect-Based Opinion Mining in Tourism Products Reviews}, In \textit{Expert Systems with Applications} Volume 41, Issue 17, 1 December 2014, Pages 7764–7775. 


\item F. Bravo-Marquez, M. Mendoza and B. Poblete \textit{Meta-Level Sentiment Models for Big Social Data Analysis}, In \textit{Knowledge-Based Systems} Volume 69, October 2014, Pages 86–99. 

\end{enumerate}


\begin{flushleft}
\emph{Conference and Workshop Papers}
\end{flushleft}

\begin{enumerate}

\item S. M. Mohammad, F. Bravo-Marquez, M. Salameh, and S. Kiritchenko  \textit{Semeval-2018 Task 1: Affect in tweets}. In \textit{Proceedings of International Workshop on Semantic Evaluation (SemEval-2018)}, New Orleans, LA, USA, June 2018. 

\item  S. M. Mohammad and F. Bravo-Marquez \textit{WASSA-2017 Shared Task on Emotion Intensity}, In \textit{Proceedings of the EMNLP 2017 Workshop on Computational Approaches to Subjectivity, Sentiment, and Social Media (WASSA)}, September 2017, Copenhagen, Denmark. 

\item S. M. Mohammad and F. Bravo-Marquez \textit{Emotion Intensities in Tweets}, In \textit{*Sem '17: Proceedings of the sixth joint conference on lexical and computational semantics (*Sem)}, August 2017, Vancouver, Canada. 


\item F. Bravo-Marquez, E. Frank, and B. Pfahringer \textit{Annotate-Sample-Average (ASA): A New Distant
Supervision Approach for Twitter Sentiment Analysis}, In \textit{The biennial European Conference  on Artificial Intelligence (ECAI'16)}. The Hague, Netherlands.


\item F. Bravo-Marquez, E. Frank, and B. Pfahringer \textit{From opinion lexicons to sentiment classification of
tweets and vice versa: a transfer learning approach}, In \textit{The IEEE/WIC/ACM International Conference on Web Intelligence (WI'16)}. Omaha, Nebraska, USA.


\item F. Bravo-Marquez, E. Frank, S. Mohammad, and B. Pfahringer \textit{Determining Word--Emotion Associations from Tweets by Multi-Label Classification}, In \textit{The IEEE/WIC/ACM International Conference on Web Intelligence (WI'16)}. Omaha, Nebraska, USA.

\item F. Bravo-Marquez, E. Frank, and B. Pfahringer \textit{Positive, Negative, or Neutral: Learning an Expanded Opinion Lexicon from Emoticon-annotated Tweets}, In \textit{IJCAI '15: Proceedings of the 24th International Joint Conference on Artificial Intelligence}. Buenos Aires, Argentina 2015.

\item F. Bravo-Marquez, E. Frank, and B. Pfahringer \textit{From Unlabelled Tweets to Twitter-specific Opinion Words}, In \textit{SIGIR '15: Proceedings of the 38th International ACM SIGIR Conference on Research \& Development in Information Retrieval}. Santiago, Chile 2015. 

\item M. Mendoza, F. Bravo-Marquez, B. Poblete, and D. Gayo-Avello \textit{Long-memory Time Series Ensembles for Concept Shift Detection} , In \textit{KDD-BigMine '13 2nd International Workshop on Big Data, Streams and Heterogeneous Source Mining: Algorithms, Systems, Programming Models and Applications}. Chicago, USA 2013.

\item F. Bravo-Marquez, M. Mendoza and B. Poblete \textit{Combining Strengths, Emotions and Polarities for Boosting Twitter Sentiment Analysis}, In \textit{KDD-WISDOM '13: 2nd Workshop on Issues of Sentiment Discovery and Opinion Mining}. Chicago, USA 2013. 

\item E. Marrese-Taylor, J.D. Velásquez, F. Bravo-Marquez \textit{OpinionZoom, a modular tool to explore tourism opinions on the Web}, \\
In \textit{WI '2013: IEEE/WIC/ACM International Conference on Web Intelligence}. Industry Track. Atlanta, USA. 

\item E. Marrese-Taylor, J.D. Velásquez, F. Bravo-Marquez, Y. Matsuo \textit{Identifying Customer Preferences about Tourism Products using an Aspect-Based Opinion Mining Approach}, In \textit{KES '13: 17th International Conference on Knowledge-Based and Intelligent Information \& Engineering Systems. Kitakyushu, Japan, 2013}. Procedia Computer Science. 

\item F. Bravo-Marquez, D. Gayo-Avello, M. Mendoza and B. Poblete \textit{Opinion Dynamics of Elections in Twitter}, In \textit{LA-WEB '12: 8th Latin American Web Congress}. Cartagena de Indias, Colombia, 2012. IEEE Computer Society's Conference Publishing Services (CPS).

\item F. Bravo-Marquez and M. Manriquez \textit{A Zipf-Like Distant Supervision Approach for Multi-Document Summarization Using Wikinews Articles}, In \textit{SPIRE '12: 19th International Symposium on String Processing and Information Retrieval}. Cartagena de Indias, Colombia, 2012. Springer-Verlag.

\item F. Bravo-Marquez, G. L'Huillier, P. Moya, S.A. Rios, and J.D. Velasquez  \textit{An Automatic Text Comprehension Classifier Based on Mental Models and Latent Semantic Features} , In \textit{I-KNOW '2011: 11th International Conference on Knowledge Management and Knowledge Technologies}. Grass, Austria. ACM ICPS.

\item F. Bravo-Marquez, G. L'Huillier, S.A. Rios, and J.D. Velasquez \textit{A Text Similarity Meta-Search Engine Based on Document Fingerprints and Search Results Records} , In \textit{WI '2011: IEEE/WIC/ACM International Conference on Web Intelligence}. Lyon, France. 

\item F. Bravo-Marquez, G. L'Huillier, S. Rios, and J.D. Velasquez  \textit{Hypergeometric Language Model and Zipf-like Scoring Function for Web Document Similarity Retrieval}, In \textit{SPIRE '10: 17th International Symposium on String Processing and Information Retrieval}. Los Cabos, Mexico, 2010. Springer-Verlag.

\item F. Bravo-Marquez, G. L'Huillier, S. Rios, J.D. Velasquez, and L. Guerrero  \textit{DOCODE-lite: A Meta-Search Engine for Document Similarity Retrieval}, In \textit{KES '10: 14th International Conference on Knowledge-Based and Intelligent Information \& Engineering Systems}. Cardiff, Wales, 2010. Springer-Verlag.

\end{enumerate}



\section{Teaching Experience}

\begin{CV}



\item [Spring 2018] (Lecturer) Practical Data Mining (COMP321), The University of Waikato

\item [June 2018] (Lecturer) Deep Learning for Natural Language Processing, IfI Summer School 2018 on Machine Learning,  Deparment of Informatics, University of Zurich. 

\item[Spring 2017] (Tutor) Practical Data Mining (COMP321), The University of Waikato.

\item[Spring 2013] (Lecturer) Databases Management Diploma, Informatic Engineering Department (postgraduate), Universidad Técnica Federico Santa María.

\item[Spring 2012] (Lecturer) Data Mining (CC5206/CC71Q), Computer Science Department (undergraduate and postgraduate), University of Chile.

\item[Fall 2011]   (Lecturer) Information Technologies and Business Process Redesign (IN72K), Master in Operations Management, University of Chile.

\item[Spring 2010]  (TA) Web Mining (IN4522), Industrial Engineering, University of Chile.

\item[Spring 2010]  (TA) ICT for globalization (IN7B0), Master in Management for Globalization, University of Chile.


\end{CV}

\section{Students}
\begin{itemize}

\item (2018) Nicole Chan, Co-supervised with Andreea Calude, ``Social Media Meets Te Reo Māori Loanwords''  Honours Project, University of Waikato. 
\item  (2018)  Joshua Lovelock, Co-supervised with Eibe Frank,``Automatic Detection of Hate Speec'', Honours Project, University of Waikato.
\item  (2018) Tristan Anderson, Co-supervised with Bernhard Pfahringer, ``Building Time-Evolving Opinion Lexicon'', Honours Project, University of Waikato.
\item (2013) Edison Marrese Taylor, Co-supervisor with Juan Velásquez, ``Diseño e Implementación de una Aplicación de Web Opinion Mining para Identificar Preferencias de Usuarios sobre Productos Turísticos de la X Región de los Lagos'', Industrial Engineering, U. of Chile.
\item (2012) Luis Maldonado, Co-supervisor with Mauricio Marín, ``Análisis de Sentimiento en el Sistema de Red Social Twitter'', Execution Informatic Engineering, U. of Santiago.

\end{itemize}


\section{Professional Community Involvement}
\begin{itemize}
  \item Co-organiser of the SemEval-2018 Task 1: Affect in Tweets.
  \item Co-organiser of the WASSA-2017 shared task on emotion intensity (EmoInt).
  \item Program committes: WASSA 2018, WISDOM 2018, IJCAI-ECAI 2018, EMNLP 2017, WASSA 2017.
  \item Journal Reviewer:  Journal of Machine Learning Research, Natural Language Engineering, IEEE Transactions on Knowledge and Data Engineering, Knowledge-based Systems, ACM Transactions on Intelligent Systems and Technology (TIST), IEEE Computational Intelligence Magazine.
 \end{itemize}
 
\section{Short Courses}
\begin{itemize}
 \item Forecasting, taught by Bruce E. Hansen at Central Bank of Chile, October 29-31, 2013. \url{http://www.ssc.wisc.edu/~bhansen/cbc/}.
 \item Machine Learning Summer School, February 16-25, 2015, Sydney, Australia. \url{http://rp-www.cs.usyd.edu.au/~mlss/}.
 \end{itemize} 
 
\section{Academic Visits}
\begin{CV}
\item [July 2018] Institute of Computational Linguistics, University of Zurich, hosted by Martin Volk.
\item [September 2017] Institute of Computational Linguistics, University of Zurich, hosted by Manfred Klenner.
\item [October 2016] National Research Council Canada (NRC), hosted by Saif Mohammad.
\end{CV} 
 
 
\section{Seminars and Talks}
\begin{CV}
\item [March 2018] Natural Language Processing in NZ Meetup. Tutorial: Using Sentiment Analysis as a Case Study for Introducing Modern NLP Concepts.

\item [February 2018] University of Waikato. Tutorial: Using Sentiment Analysis as a Case Study for Introducing Modern NLP Concepts.

\item [January 2018]Universidad de Chile. Tutorial: Using Sentiment Analysis as a Case Study for Introducing Modern NLP Concepts.

\item [January 2018] Pontifica Universidad Catolica de Chile. Invited talk at CIWS Workshop - Future of Data: Emotion Intensties of Tweets.

\item [September 2017] Institute of Computational Linguistics, University of Zurich:  Acquiring and Exploiting Lexical Knowledge for Twitter Sentiment Analysis.

\item [October 2016] National Research Council Canada (NRC):  Acquiring and Exploiting Lexical Knowledge for Twitter Sentiment Analysis.

\item [September 2016] University of Melbourne, hosted by Timothy Baldwin: Acquiring and Exploiting Lexical Knowledge for Twitter Sentiment Analysis.

\item [July 2016] Pontifica Universidad Catolica de Chile:  Acquiring and Exploiting Lexical Knowledge for Twitter Sentiment Analysis.

\item [July 2016] Universidad de Chile:  Acquiring and Exploiting Lexical Knowledge for Twitter Sentiment Analysis.

\item [June 2016] Universidad Técnica Federico Santa María: Acquiring and Exploiting Lexical Knowledge for Twitter Sentiment Analysis. 
\end{CV}  

\section{Technical Skills}
\begin{flushleft}
\textbf{Programming Languages}: Java, R, Python, C,C++. \\
\textbf{Database Systems}: PostgreSQL, MySQL, MongoDB.\\
\end{flushleft}

\section{Languages}
\begin{flushleft}
\textbf{Spanish}: Native.  \\
\textbf{English}: Fluent, \textit{TOEFL IBT 103/120}.   \\
\textbf{German}:  Average, \textit{Sprachdiplom Zweite Stufe}. \\ 
\end{flushleft}

\section{References}
\begin{flushleft}
\textbf{Bernhard Pfahringer} \\
Full Professor, University of Waikato. \\
\textbf{Eibe Frank} \\
Associate Professor, University of Waikato. \\
\textbf{Saif Mohammad}   \\
Senior Research Officer, National Research Council Canada (NRC). \\ 
\end{flushleft}


%\vspace{\baselineskip}
\begin{flushright}
Felipe José Bravo Márquez\\
Hamilton, New Zealand \today
\end{flushright}

\end{document}
